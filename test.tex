\documentclass[10pt]{extarticle}
%\documentclass[]{file}
\usepackage[margin=0.2in]{geometry}
\usepackage{romannum}
\usepackage[most]{tcolorbox}
\usepackage{enumitem}
\usepackage{fontawesome}
\usepackage{hyperref}
\usepackage{tabularx}
\usepackage{multicol}
\usepackage{tfrupee}

\setlist[itemize]{noitemsep, topsep=0pt}
\addtolength{\parskip}{-2mm}
\tcbset{
    frame code={}
    center title,
    left=0pt,
    right=0pt,
    top=0pt,
    bottom=0pt,
    colback=gray!50,
    colframe=white,
    width=\dimexpr\textwidth\relax,
    enlarge left by=0mm,
    boxsep=3pt,
    arc=0pt,outer arc=0pt,
    }
\urlstyle{same}

\raggedright
\setlength{\tabcolsep}{0in}
\begin{document}
\begin{flushleft}
\noindent {\huge\textbf{Lavesh Mangal}}
\end{flushleft}
%3rd Year Undergraduate  \hfill\textbf{Email : }laveshm20@iitk.ac.in
%\\Mathematics \& Scientific Computing \hfill\textbf{Phone : }+91-8078672915
4th Year Undergraduate Student \hfill \faEnvelope\
\href{mailto:laveshm20@iitk.ac.in}{ laveshm20@iitk.ac.in }$|$ \faGithub\ \href{https://github.com/LaveshM}{LaveshM\ }
\\Mathematics and Scientific Computing \hfill \faMobile\ (+91)8078672915 $|$ \faLinkedin\ \href{https://www.linkedin.com/in/lavesh-mangal-620850212/}{Lavesh Mangal}
%\\\textbf{Phone : }+91-8078672915 $|$ \textbf{Email : }\href{mailto:laveshm20@iitk.ac.in}{laveshm20@iitk.ac.in} $|$ \faGithub\ \href{https://github.com/LaveshM}{LaveshM} $|$ \faLinkedin\ \href{https://www.linkedin.com/in/lavesh-mangal-620850212/}{Lavesh Mangal} 
\vspace{-6pt}
\\
\noindent\rule[0.5ex]{\linewidth}{1pt}
\vspace{-8.5mm}
{\large \textbf{\begin{tcolorbox}\textsc{Academic Qualifications}\end{tcolorbox}}}
\vspace{-4.5mm}

\setlength{\tabcolsep}{19pt}
\begin{center}
\begin{tabularx}
{\dimexpr\textwidth-5mm\relax}{|c|c|c|c|}
\hline
\textbf{Year} & \textbf{Degree/Certificate } & \textbf{Institute} & \textbf{CPI/Percentage} \\
\hline
2020 - Present & B.S. - M.S. & Indian Institute of Technology, Kanpur & 8.6/10\\
\hline
2020 & CBSE(\Romannum{12}) & Anand Vidya Mandir, Bharatpur & 92.4$\%$ \\
\hline
2018 & RBSE(\Romannum{10}) & Little Flower Secondary School, Bayana & 88.0$\%$ \\
\hline
\end{tabularx}
\end{center}
\vspace{-1mm}
%\begin{center}
%\begin{tabular}{|p{2.5cm}|p{6.0cm}|p{8.0cm}|p{2.2cm}|}
%\hline
%\centering{\textbf{Year}} & \centering{\textbf{Degree/Certificate}} & \centering{\textbf{Institute}} & \centering{\textbf{CPI/$\%$}} \\
%\hline
%\\\centering{2020 - Present} & \centering{B.S.} & \centering{Indian Institute of Technology, Kanpur} & \centering{7.4/10} \\
%\hline
%\\\centering{2020} & \centering{CBSE(\Romannum{12})} & \centering{Anand Vidya Mandir, Bharatpur} & \centering{92.4$\%$} \\
%\hline
%\\\centering{2018} & \centering{RBSE(\Romannum{10})} & \centering{Little Flower Secondary School, Bayana} & \centering{88.0$\%$} \\
%\hline
%\end{tabular}
%\end{center}
{\large \textbf{\begin{tcolorbox}\textsc{Scholastic Achievements}\end{tcolorbox}}}
\vspace{-4mm}
\begin{itemize}
\item Awarded with the \textbf{Academic Excellence Award} for the year 2022-23 for being in top 10\% of the batch.
\item \textbf{Subject Topper} of Mathematics in \textbf{Mimamsa 2023}, a national-level science exam conducted by IISER Pune.
\item Received \textbf{Outstanding (A*) Grade} in the courses \textbf{Probability and Statistics} and \textbf{Image Processing}.
\item Secured \textbf{All India Rank 815} in \textbf{JEE Advanced 2020} among the \textbf{150,000} shortlisted candidates.
\end{itemize}
\medskip
{\large \textbf{\begin{tcolorbox}\textsc{Work Experience}\end{tcolorbox}}}
\vspace{-4mm}
\begin{itemize}

\item \textbf{VISA} $|$ \textit{\small Summer Software Intern in Cybersecurity Team} \hfill\hfill(\textit{May'23 - Jul'23})\\
\begin{itemize}[leftmargin=2mm]
\item Received \textbf{Pre-Placement Offer} (PPO) for excellent performance during internship.
\item Developed over 40 REST API Python scripts using requests package to automate repetitive tasks and do complex queries.
\item Worked on designing and developing web pages using Angular framework of JavaScript.
\end{itemize}


\end{itemize}
\vspace{0mm}
{\large \textbf{\begin{tcolorbox}\textsc{Key Projects}\end{tcolorbox}}}
\vspace{-4mm}
\begin{itemize} 

\item \textbf{Meta Bayesian Learning for Channel Estimation} (Undergraduate Project) $|$ \textit{\small{Prof. Rohit Budhiraja}} \hfill\hfill(\textit{Jul'23 -  Nov'23})\\
\begin{itemize}[leftmargin=2mm]

\item Reviewed “Sparse Bayesian Learning and the Relevance Vector Machine” and implemented its code for EM updates.
\item Studied research papers to learn meta learning techniques such as \textbf{MAML}, FOMAML, etc. and their implementations.
\item Extended meta learning to bayesian framework such as Variational Agnostic Modelling that Performs Inference for Robust Estimation (VAMPIRE) for the demodulation of signal in IoT scenario.
\end{itemize}

\item \textbf{Solving the Minigrid} (Course Project) $|$ \textit{\small{Course : EE675 - Reinforcement Learning}} $|$\textit{\small{Prof. S. Swamy Peruru}}\hfill\hfill(\textit{Jan'23 -  Apr'23})\\
\begin{itemize}[leftmargin=2mm]
% \item Implemented \textbf{Policy Gradient REINFORCE} for solving the Minigrid-DoorKey5x5 environment of gymnasium.
% \item Implemented \textbf{Proximal Policy Optimisation (PPO)} for solving the 8x8 variant of the same environment. 
% \item Major steps included were setting up the environment, changing the reward mechanism and writing a report.
\item Employed \textbf{Policy Gradient REINFORCE} method for Minigrid-DoorKey environment in Gymnasium framework.
\item Developed an \textbf{actor-critic network} for \textbf{Proximal Policy Optimisation (PPO)} and integrating a CNN network to it, achieving \textbf{100\% convergence in 90k iterations} for 8x8 variant of the environment.

\end{itemize}

\item \textbf{Escaping the Caves} (Course Project) $|$ \textit{\small{Course : CS641 - Modern Cryptology}} $|$ \textit{\small{Prof. Manindra Agrawal}} \hfill\hfill(\textit{Jan'23 -  Apr'23})\\
\begin{itemize}[leftmargin=2mm]
% \item Implemented and did the \textbf{Crypt-analysis} of existing \textbf{Cryptography systems} and their security by decrypting ciphertexts.
\item Programmed decryption of \textbf{6-round DES, Vigenere, EAEAE(AES), Permutation and Caesar Shift Cipher}.
\item Wrote detailed report on each and achieved top scores in the class leading to A grade in the course.
\end{itemize}

\item \textbf{Object Detection and Image Recognition on Large Images} $|$ \textit{\small{Prof. Tushar Sandhan}}\hfill\hfill(\textit{May'22 - Jun'22})
%\textbf{Mentor: Prof. Tushar Sandhan}, Assistant Professor, Department of Electrical Engineering, IIT Kanpur
\begin{itemize}[leftmargin=2mm]
\item Implemented transfer learning on the \textbf{YOLOv5} algorithm and attained an accuracy of \textbf{67.3\%} on our custom data set.
\item Worked on \textbf{Convolution Neural Network} backbone VGG16 model for extracting feature maps of images.
% \item Worked on our custom detector using SSD, which works on large-size images along with minimising down-sampling losses.
\end{itemize}

\item \textbf{Markov Chain Monte Carlo Methods using Julia}, \textit{\small Society of Statistics and Mathematics, IIT Kanpur} \href{https://github.com/LaveshM/mcmc-methods-in-julia-1}{\faGithub}\hfill\hfill(\textit{Apr'22 - Jul'22})
\begin{itemize}[leftmargin=2mm]
\item Implemented \textbf{Pseudo-random generators (PRG)} and discrete \textbf{Inverse Transform} and accept-reject samplers.
% \item Used \textbf{Accept-reject} algorithm to sample from p-dimensional \textbf{hyper-sphere} and \textbf{truncated normal} distributions.
\item Implemented the MCMC \textbf{Metropolis-Hastings} algorithm to sample from several complex posterior distributions.
\item Studied \textbf{Stochastic Gradient MCMC} and examined the code of \textbf{Bayesian Neural Network} using SGMCMC.
\item Programmed \textbf{Bernoulli-factory} based MCMC algorithms, which are computationally more efficient than accept-reject.
\end{itemize}

\end{itemize}

\vspace{-8mm}

\medskip
{\large \textbf{\begin{tcolorbox}\textsc{Technical Skills}\end{tcolorbox}}}
\vspace{-4mm}
\begin{center}
\begin{tabular}{|p{5.1cm}|p{3.7cm}|p{7.5cm}|}
\hline
\textbf{Programming Languages}  & \textbf{Software and Tools} & \textbf{Python Libraries }\\
\hline
C, C++, MATLAB, Python   &  Premiere Pro, MS Office & NumPy, Pandas, PyTorch, OpenCV, Requests\\
\hline
\end{tabular}
\end{center}
%\hfill$*_{Ongoing\ Courses}$
{\large \begin{tcolorbox}\textsc{\textbf{Relevant Courses}}\end{tcolorbox}}
\vspace{-4mm}
\begin{center}
\begin{tabular}{|p{4.4cm}|p{4.3cm}|p{7.6cm}|}
\hline
Set Theory \& Logic & Probability and Statistics & Applied Stochastic Process \\
Linear Algebra & Abstract Algebra & Modern Cryptology \\
Fundamental of Computing &  Data Structure \& Algorithm & Numerical Analysis and Scientific Computing \\
Real Analysis & Complex Analysis & Several Variable Calculus \& Differential Geometry\\
Reinforcement Learning & Image Processing & Introduction to Machine Learning \\
\hline
\end{tabular}
\end{center}
%\medskip
%\newpage
%\end{itemize}
{\large \textbf{\begin{tcolorbox}\textsc{Extra-Curricular Activities}\end{tcolorbox}}}
\vspace{-4mm}
\begin{itemize}
\item Made a video essay on a movie while working in a team with guidance from \textbf{Film Club}, IIT Kanpur, in my first year.
\item Organised workshops and taught \textbf{origami} to \textbf{15+ freshers} and \textbf{30+ school students} and received positive feedback.
\item Worked as coordinator of Vivekananda Samiti and organised cloth and blanket donation drives to the nearby slums and villages.

\end{itemize}
\end{document}

